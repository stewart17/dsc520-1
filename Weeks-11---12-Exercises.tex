% Options for packages loaded elsewhere
\PassOptionsToPackage{unicode}{hyperref}
\PassOptionsToPackage{hyphens}{url}
%
\documentclass[
]{article}
\usepackage{amsmath,amssymb}
\usepackage{lmodern}
\usepackage{iftex}
\ifPDFTeX
  \usepackage[T1]{fontenc}
  \usepackage[utf8]{inputenc}
  \usepackage{textcomp} % provide euro and other symbols
\else % if luatex or xetex
  \usepackage{unicode-math}
  \defaultfontfeatures{Scale=MatchLowercase}
  \defaultfontfeatures[\rmfamily]{Ligatures=TeX,Scale=1}
\fi
% Use upquote if available, for straight quotes in verbatim environments
\IfFileExists{upquote.sty}{\usepackage{upquote}}{}
\IfFileExists{microtype.sty}{% use microtype if available
  \usepackage[]{microtype}
  \UseMicrotypeSet[protrusion]{basicmath} % disable protrusion for tt fonts
}{}
\makeatletter
\@ifundefined{KOMAClassName}{% if non-KOMA class
  \IfFileExists{parskip.sty}{%
    \usepackage{parskip}
  }{% else
    \setlength{\parindent}{0pt}
    \setlength{\parskip}{6pt plus 2pt minus 1pt}}
}{% if KOMA class
  \KOMAoptions{parskip=half}}
\makeatother
\usepackage{xcolor}
\IfFileExists{xurl.sty}{\usepackage{xurl}}{} % add URL line breaks if available
\IfFileExists{bookmark.sty}{\usepackage{bookmark}}{\usepackage{hyperref}}
\hypersetup{
  pdftitle={Weeks 11 \& 12 Exercises},
  pdfauthor={Stewart Wilson},
  hidelinks,
  pdfcreator={LaTeX via pandoc}}
\urlstyle{same} % disable monospaced font for URLs
\usepackage[margin=1in]{geometry}
\usepackage{color}
\usepackage{fancyvrb}
\newcommand{\VerbBar}{|}
\newcommand{\VERB}{\Verb[commandchars=\\\{\}]}
\DefineVerbatimEnvironment{Highlighting}{Verbatim}{commandchars=\\\{\}}
% Add ',fontsize=\small' for more characters per line
\usepackage{framed}
\definecolor{shadecolor}{RGB}{248,248,248}
\newenvironment{Shaded}{\begin{snugshade}}{\end{snugshade}}
\newcommand{\AlertTok}[1]{\textcolor[rgb]{0.94,0.16,0.16}{#1}}
\newcommand{\AnnotationTok}[1]{\textcolor[rgb]{0.56,0.35,0.01}{\textbf{\textit{#1}}}}
\newcommand{\AttributeTok}[1]{\textcolor[rgb]{0.77,0.63,0.00}{#1}}
\newcommand{\BaseNTok}[1]{\textcolor[rgb]{0.00,0.00,0.81}{#1}}
\newcommand{\BuiltInTok}[1]{#1}
\newcommand{\CharTok}[1]{\textcolor[rgb]{0.31,0.60,0.02}{#1}}
\newcommand{\CommentTok}[1]{\textcolor[rgb]{0.56,0.35,0.01}{\textit{#1}}}
\newcommand{\CommentVarTok}[1]{\textcolor[rgb]{0.56,0.35,0.01}{\textbf{\textit{#1}}}}
\newcommand{\ConstantTok}[1]{\textcolor[rgb]{0.00,0.00,0.00}{#1}}
\newcommand{\ControlFlowTok}[1]{\textcolor[rgb]{0.13,0.29,0.53}{\textbf{#1}}}
\newcommand{\DataTypeTok}[1]{\textcolor[rgb]{0.13,0.29,0.53}{#1}}
\newcommand{\DecValTok}[1]{\textcolor[rgb]{0.00,0.00,0.81}{#1}}
\newcommand{\DocumentationTok}[1]{\textcolor[rgb]{0.56,0.35,0.01}{\textbf{\textit{#1}}}}
\newcommand{\ErrorTok}[1]{\textcolor[rgb]{0.64,0.00,0.00}{\textbf{#1}}}
\newcommand{\ExtensionTok}[1]{#1}
\newcommand{\FloatTok}[1]{\textcolor[rgb]{0.00,0.00,0.81}{#1}}
\newcommand{\FunctionTok}[1]{\textcolor[rgb]{0.00,0.00,0.00}{#1}}
\newcommand{\ImportTok}[1]{#1}
\newcommand{\InformationTok}[1]{\textcolor[rgb]{0.56,0.35,0.01}{\textbf{\textit{#1}}}}
\newcommand{\KeywordTok}[1]{\textcolor[rgb]{0.13,0.29,0.53}{\textbf{#1}}}
\newcommand{\NormalTok}[1]{#1}
\newcommand{\OperatorTok}[1]{\textcolor[rgb]{0.81,0.36,0.00}{\textbf{#1}}}
\newcommand{\OtherTok}[1]{\textcolor[rgb]{0.56,0.35,0.01}{#1}}
\newcommand{\PreprocessorTok}[1]{\textcolor[rgb]{0.56,0.35,0.01}{\textit{#1}}}
\newcommand{\RegionMarkerTok}[1]{#1}
\newcommand{\SpecialCharTok}[1]{\textcolor[rgb]{0.00,0.00,0.00}{#1}}
\newcommand{\SpecialStringTok}[1]{\textcolor[rgb]{0.31,0.60,0.02}{#1}}
\newcommand{\StringTok}[1]{\textcolor[rgb]{0.31,0.60,0.02}{#1}}
\newcommand{\VariableTok}[1]{\textcolor[rgb]{0.00,0.00,0.00}{#1}}
\newcommand{\VerbatimStringTok}[1]{\textcolor[rgb]{0.31,0.60,0.02}{#1}}
\newcommand{\WarningTok}[1]{\textcolor[rgb]{0.56,0.35,0.01}{\textbf{\textit{#1}}}}
\usepackage{graphicx}
\makeatletter
\def\maxwidth{\ifdim\Gin@nat@width>\linewidth\linewidth\else\Gin@nat@width\fi}
\def\maxheight{\ifdim\Gin@nat@height>\textheight\textheight\else\Gin@nat@height\fi}
\makeatother
% Scale images if necessary, so that they will not overflow the page
% margins by default, and it is still possible to overwrite the defaults
% using explicit options in \includegraphics[width, height, ...]{}
\setkeys{Gin}{width=\maxwidth,height=\maxheight,keepaspectratio}
% Set default figure placement to htbp
\makeatletter
\def\fps@figure{htbp}
\makeatother
\setlength{\emergencystretch}{3em} % prevent overfull lines
\providecommand{\tightlist}{%
  \setlength{\itemsep}{0pt}\setlength{\parskip}{0pt}}
\setcounter{secnumdepth}{-\maxdimen} % remove section numbering
\ifLuaTeX
  \usepackage{selnolig}  % disable illegal ligatures
\fi

\title{Weeks 11 \& 12 Exercises}
\author{Stewart Wilson}
\date{2022-06-02}

\begin{document}
\maketitle

\hypertarget{introduction-to-machine-learning}{%
\section{Introduction to Machine
Learning}\label{introduction-to-machine-learning}}

\hypertarget{scatter-plot-of-data}{%
\subsection{Scatter Plot of Data}\label{scatter-plot-of-data}}

\includegraphics{Weeks-11---12-Exercises_files/figure-latex/unnamed-chunk-1-1.pdf}
\includegraphics{Weeks-11---12-Exercises_files/figure-latex/unnamed-chunk-1-2.pdf}
\#\# Training Binary Data Sets

\begin{Shaded}
\begin{Highlighting}[]
\CommentTok{\# split binary into test and training set}
\NormalTok{train\_index }\OtherTok{\textless{}{-}} \FunctionTok{createDataPartition}\NormalTok{(binary}\SpecialCharTok{$}\NormalTok{label, }\AttributeTok{p=}\NormalTok{.}\DecValTok{6}\NormalTok{)}\SpecialCharTok{$}\NormalTok{Resample1}
\NormalTok{training\_binary }\OtherTok{\textless{}{-}}\NormalTok{ binary[train\_index, ]}
\NormalTok{test\_binary }\OtherTok{\textless{}{-}}\NormalTok{ binary[}\SpecialCharTok{{-}}\NormalTok{train\_index, ]}
\CommentTok{\# matrix for num of clusters}
\NormalTok{kmatrix }\OtherTok{\textless{}{-}} \FunctionTok{c}\NormalTok{(}\DecValTok{3}\NormalTok{, }\DecValTok{5}\NormalTok{, }\DecValTok{10}\NormalTok{, }\DecValTok{15}\NormalTok{, }\DecValTok{20}\NormalTok{, }\DecValTok{25}\NormalTok{)}
\CommentTok{\# will collect accuracy count for clusters}
\NormalTok{accBin }\OtherTok{\textless{}{-}} \FunctionTok{c}\NormalTok{()}
\CommentTok{\# runs nearest neighbor for every given k in kmatrix}
\CommentTok{\# calculates accuracy for each loops}
\NormalTok{index }\OtherTok{\textless{}{-}} \DecValTok{0}
\ControlFlowTok{for}\NormalTok{(i }\ControlFlowTok{in}\NormalTok{ kmatrix)\{}
\NormalTok{  index }\OtherTok{=}\NormalTok{ i }\SpecialCharTok{+} \DecValTok{1}
\NormalTok{  kModBin }\OtherTok{\textless{}{-}} \FunctionTok{knn}\NormalTok{(}\AttributeTok{train=}\NormalTok{training\_binary, }\AttributeTok{test=}\NormalTok{test\_binary, }\AttributeTok{cl=}\NormalTok{training\_binary}\SpecialCharTok{$}\NormalTok{label, }\AttributeTok{k=}\NormalTok{i)}
\NormalTok{  accBin[index] }\OtherTok{\textless{}{-}} \DecValTok{100}\SpecialCharTok{*}\FunctionTok{sum}\NormalTok{(test\_binary}\SpecialCharTok{$}\NormalTok{label }\SpecialCharTok{==}\NormalTok{ kModBin)}\SpecialCharTok{/}\FunctionTok{NROW}\NormalTok{(test\_binary}\SpecialCharTok{$}\NormalTok{label)}
\NormalTok{\}}
\CommentTok{\# plotting accuracy vs k}
\NormalTok{accuracy\_df }\OtherTok{\textless{}{-}} \FunctionTok{data.frame}\NormalTok{(accBin)}
\NormalTok{accuracy\_df}\SpecialCharTok{$}\NormalTok{k }\OtherTok{\textless{}{-}} \FunctionTok{seq.int}\NormalTok{(}\FunctionTok{nrow}\NormalTok{(accuracy\_df))}
\NormalTok{accuracy\_plot }\OtherTok{\textless{}{-}} \FunctionTok{ggplot}\NormalTok{(}\AttributeTok{data =}\NormalTok{ accuracy\_df, }\FunctionTok{aes}\NormalTok{(}\AttributeTok{x =}\NormalTok{ k, }\AttributeTok{y =}\NormalTok{ accBin)) }\SpecialCharTok{+} \FunctionTok{geom\_point}\NormalTok{() }\SpecialCharTok{+} \FunctionTok{xlab}\NormalTok{(}\StringTok{"k"}\NormalTok{) }\SpecialCharTok{+} \FunctionTok{ylab}\NormalTok{(}\StringTok{"Accuracy"}\NormalTok{) }\SpecialCharTok{+} \FunctionTok{ggtitle}\NormalTok{(}\StringTok{"Binary Cluster Accuracy"}\NormalTok{)}
\NormalTok{accuracy\_plot}
\end{Highlighting}
\end{Shaded}

\begin{verbatim}
## Warning: Removed 20 rows containing missing values (geom_point).
\end{verbatim}

\includegraphics{Weeks-11---12-Exercises_files/figure-latex/unnamed-chunk-2-1.pdf}
\#\# Training Trinary Data Sets

\begin{Shaded}
\begin{Highlighting}[]
\CommentTok{\# split trinary into test and training set}
\NormalTok{train\_index }\OtherTok{\textless{}{-}} \FunctionTok{createDataPartition}\NormalTok{(trinary}\SpecialCharTok{$}\NormalTok{label, }\AttributeTok{p=}\NormalTok{.}\DecValTok{6}\NormalTok{)}\SpecialCharTok{$}\NormalTok{Resample1}
\NormalTok{training\_trinary }\OtherTok{\textless{}{-}}\NormalTok{ trinary[train\_index, ]}
\NormalTok{test\_trinary }\OtherTok{\textless{}{-}}\NormalTok{ trinary[}\SpecialCharTok{{-}}\NormalTok{train\_index, ]}
\CommentTok{\# matrix for num of clusters}
\NormalTok{kmatrix }\OtherTok{\textless{}{-}} \FunctionTok{c}\NormalTok{(}\DecValTok{3}\NormalTok{, }\DecValTok{5}\NormalTok{, }\DecValTok{10}\NormalTok{, }\DecValTok{15}\NormalTok{, }\DecValTok{20}\NormalTok{, }\DecValTok{25}\NormalTok{)}
\CommentTok{\# will collect accuracy count for clusters}
\NormalTok{acc\_Trin }\OtherTok{\textless{}{-}} \FunctionTok{c}\NormalTok{()}
\CommentTok{\# runs nearest neighbor for every given k in kmatrix}
\CommentTok{\# calculates accuracy for each loops}
\NormalTok{index }\OtherTok{\textless{}{-}} \DecValTok{0}
\ControlFlowTok{for}\NormalTok{(i }\ControlFlowTok{in}\NormalTok{ kmatrix)\{}
\NormalTok{  index }\OtherTok{=}\NormalTok{ i }\SpecialCharTok{+} \DecValTok{1}
\NormalTok{  kModBin }\OtherTok{\textless{}{-}} \FunctionTok{knn}\NormalTok{(}\AttributeTok{train=}\NormalTok{training\_trinary, }\AttributeTok{test=}\NormalTok{test\_trinary, }\AttributeTok{cl=}\NormalTok{training\_trinary}\SpecialCharTok{$}\NormalTok{label, }\AttributeTok{k=}\NormalTok{i)}
\NormalTok{  acc\_Trin[index] }\OtherTok{\textless{}{-}} \DecValTok{100}\SpecialCharTok{*}\FunctionTok{sum}\NormalTok{(test\_trinary}\SpecialCharTok{$}\NormalTok{label }\SpecialCharTok{==}\NormalTok{ kModBin)}\SpecialCharTok{/}\FunctionTok{NROW}\NormalTok{(test\_trinary}\SpecialCharTok{$}\NormalTok{label)}
\NormalTok{\}}
\CommentTok{\# plotting accuracy vs k}
\NormalTok{accuracy\_df2 }\OtherTok{\textless{}{-}} \FunctionTok{data.frame}\NormalTok{(acc\_Trin)}
\NormalTok{accuracy\_df2}\SpecialCharTok{$}\NormalTok{k }\OtherTok{\textless{}{-}} \FunctionTok{seq.int}\NormalTok{(}\FunctionTok{nrow}\NormalTok{(accuracy\_df2))}
\NormalTok{accuracy\_plot2 }\OtherTok{\textless{}{-}} \FunctionTok{ggplot}\NormalTok{(}\AttributeTok{data =}\NormalTok{ accuracy\_df2, }\FunctionTok{aes}\NormalTok{(}\AttributeTok{x =}\NormalTok{ k, }\AttributeTok{y =}\NormalTok{ acc\_Trin)) }\SpecialCharTok{+} \FunctionTok{geom\_point}\NormalTok{() }\SpecialCharTok{+} \FunctionTok{ylab}\NormalTok{(}\StringTok{"Accuracy"}\NormalTok{) }\SpecialCharTok{+} \FunctionTok{ggtitle}\NormalTok{(}\StringTok{"Trinary Cluster Accuracy"}\NormalTok{)}
\NormalTok{accuracy\_plot2}
\end{Highlighting}
\end{Shaded}

\begin{verbatim}
## Warning: Removed 20 rows containing missing values (geom_point).
\end{verbatim}

\includegraphics{Weeks-11---12-Exercises_files/figure-latex/unnamed-chunk-3-1.pdf}
\#\#\# Linear Model? Looking back at the original graphs of the data, I
do not think a linear classifier would work well on these datasets.
There is no clear straight line that could split either dataset nicely
so the model would most likely not be very accurate.

Looking back at the accuracy from last week's exercises (58\%), it is
clear that clustering vastly improved the accuracy of the model.

\hypertarget{clustering}{%
\section{Clustering}\label{clustering}}

\begin{Shaded}
\begin{Highlighting}[]
\NormalTok{clustering }\OtherTok{\textless{}{-}} \FunctionTok{read.csv}\NormalTok{(}\StringTok{"C:/Users/stewa/Documents/GitHub/dsc520{-}1/data/clustering{-}data.csv"}\NormalTok{)}
\CommentTok{\# scatterplot of the data}
\NormalTok{clust\_scatter }\OtherTok{\textless{}{-}} \FunctionTok{ggplot}\NormalTok{(}\AttributeTok{data=}\NormalTok{clustering, }\FunctionTok{aes}\NormalTok{(}\AttributeTok{x=}\NormalTok{x, }\AttributeTok{y=}\NormalTok{y)) }\SpecialCharTok{+} \FunctionTok{geom\_point}\NormalTok{()}
\NormalTok{clust\_scatter}
\end{Highlighting}
\end{Shaded}

\includegraphics{Weeks-11---12-Exercises_files/figure-latex/unnamed-chunk-4-1.pdf}

\begin{Shaded}
\begin{Highlighting}[]
\CommentTok{\# set seeds and plots each cluster result for every k between 2 and 12}
\FunctionTok{set.seed}\NormalTok{(}\DecValTok{278613}\NormalTok{)}
\NormalTok{km2 }\OtherTok{\textless{}{-}} \FunctionTok{kmeans}\NormalTok{(}\AttributeTok{x=}\NormalTok{clustering, }\AttributeTok{centers =} \DecValTok{2}\NormalTok{)}
\FunctionTok{plot}\NormalTok{(km2, }\AttributeTok{data=}\NormalTok{clustering)}
\end{Highlighting}
\end{Shaded}

\includegraphics{Weeks-11---12-Exercises_files/figure-latex/unnamed-chunk-5-1.pdf}

\begin{Shaded}
\begin{Highlighting}[]
\NormalTok{km3 }\OtherTok{\textless{}{-}}  \FunctionTok{kmeans}\NormalTok{(}\AttributeTok{x=}\NormalTok{clustering, }\AttributeTok{centers =} \DecValTok{3}\NormalTok{)}
\FunctionTok{plot}\NormalTok{(km3, }\AttributeTok{data=}\NormalTok{clustering)}
\end{Highlighting}
\end{Shaded}

\includegraphics{Weeks-11---12-Exercises_files/figure-latex/unnamed-chunk-5-2.pdf}

\begin{Shaded}
\begin{Highlighting}[]
\NormalTok{km4 }\OtherTok{\textless{}{-}}  \FunctionTok{kmeans}\NormalTok{(}\AttributeTok{x=}\NormalTok{clustering, }\AttributeTok{centers =} \DecValTok{4}\NormalTok{)}
\FunctionTok{plot}\NormalTok{(km4, }\AttributeTok{data=}\NormalTok{clustering)}
\end{Highlighting}
\end{Shaded}

\includegraphics{Weeks-11---12-Exercises_files/figure-latex/unnamed-chunk-5-3.pdf}

\begin{Shaded}
\begin{Highlighting}[]
\NormalTok{km5 }\OtherTok{\textless{}{-}}  \FunctionTok{kmeans}\NormalTok{(}\AttributeTok{x=}\NormalTok{clustering, }\AttributeTok{centers =} \DecValTok{5}\NormalTok{)}
\FunctionTok{plot}\NormalTok{(km5, }\AttributeTok{data=}\NormalTok{clustering)}
\end{Highlighting}
\end{Shaded}

\includegraphics{Weeks-11---12-Exercises_files/figure-latex/unnamed-chunk-5-4.pdf}

\begin{Shaded}
\begin{Highlighting}[]
\NormalTok{km6 }\OtherTok{\textless{}{-}}  \FunctionTok{kmeans}\NormalTok{(}\AttributeTok{x=}\NormalTok{clustering, }\AttributeTok{centers =} \DecValTok{6}\NormalTok{)}
\FunctionTok{plot}\NormalTok{(km6, }\AttributeTok{data=}\NormalTok{clustering)}
\end{Highlighting}
\end{Shaded}

\includegraphics{Weeks-11---12-Exercises_files/figure-latex/unnamed-chunk-5-5.pdf}

\begin{Shaded}
\begin{Highlighting}[]
\NormalTok{km7 }\OtherTok{\textless{}{-}}  \FunctionTok{kmeans}\NormalTok{(}\AttributeTok{x=}\NormalTok{clustering, }\AttributeTok{centers =} \DecValTok{7}\NormalTok{)}
\FunctionTok{plot}\NormalTok{(km7, }\AttributeTok{data=}\NormalTok{clustering)}
\end{Highlighting}
\end{Shaded}

\includegraphics{Weeks-11---12-Exercises_files/figure-latex/unnamed-chunk-5-6.pdf}

\begin{Shaded}
\begin{Highlighting}[]
\NormalTok{km8 }\OtherTok{\textless{}{-}}  \FunctionTok{kmeans}\NormalTok{(}\AttributeTok{x=}\NormalTok{clustering, }\AttributeTok{centers =} \DecValTok{8}\NormalTok{)}
\FunctionTok{plot}\NormalTok{(km8, }\AttributeTok{data=}\NormalTok{clustering)}
\end{Highlighting}
\end{Shaded}

\includegraphics{Weeks-11---12-Exercises_files/figure-latex/unnamed-chunk-5-7.pdf}

\begin{Shaded}
\begin{Highlighting}[]
\NormalTok{km9 }\OtherTok{\textless{}{-}}  \FunctionTok{kmeans}\NormalTok{(}\AttributeTok{x=}\NormalTok{clustering, }\AttributeTok{centers =} \DecValTok{9}\NormalTok{)}
\FunctionTok{plot}\NormalTok{(km9, }\AttributeTok{data=}\NormalTok{clustering)}
\end{Highlighting}
\end{Shaded}

\includegraphics{Weeks-11---12-Exercises_files/figure-latex/unnamed-chunk-5-8.pdf}

\begin{Shaded}
\begin{Highlighting}[]
\NormalTok{km10 }\OtherTok{\textless{}{-}} \FunctionTok{kmeans}\NormalTok{(}\AttributeTok{x=}\NormalTok{clustering, }\AttributeTok{centers =}\DecValTok{10}\NormalTok{)}
\FunctionTok{plot}\NormalTok{(km10, }\AttributeTok{data=}\NormalTok{clustering)}
\end{Highlighting}
\end{Shaded}

\includegraphics{Weeks-11---12-Exercises_files/figure-latex/unnamed-chunk-5-9.pdf}

\begin{Shaded}
\begin{Highlighting}[]
\NormalTok{km11 }\OtherTok{\textless{}{-}}  \FunctionTok{kmeans}\NormalTok{(}\AttributeTok{x=}\NormalTok{clustering, }\AttributeTok{centers =} \DecValTok{11}\NormalTok{)}
\FunctionTok{plot}\NormalTok{(km11, }\AttributeTok{data=}\NormalTok{clustering)}
\end{Highlighting}
\end{Shaded}

\includegraphics{Weeks-11---12-Exercises_files/figure-latex/unnamed-chunk-5-10.pdf}

\begin{Shaded}
\begin{Highlighting}[]
\NormalTok{km12 }\OtherTok{\textless{}{-}}  \FunctionTok{kmeans}\NormalTok{(}\AttributeTok{x=}\NormalTok{clustering, }\AttributeTok{centers =} \DecValTok{12}\NormalTok{)}
\FunctionTok{plot}\NormalTok{(km12, }\AttributeTok{data=}\NormalTok{clustering)}
\end{Highlighting}
\end{Shaded}

\includegraphics{Weeks-11---12-Exercises_files/figure-latex/unnamed-chunk-5-11.pdf}
\#\# Average Distance

\begin{Shaded}
\begin{Highlighting}[]
\CommentTok{\# list of distances for each k}
\NormalTok{distances }\OtherTok{\textless{}{-}} \DecValTok{2}\SpecialCharTok{:}\DecValTok{12} \SpecialCharTok{\%\textgreater{}\%} \FunctionTok{map}\NormalTok{(}\ControlFlowTok{function}\NormalTok{(k) }\FunctionTok{kmeans}\NormalTok{(}\AttributeTok{x=}\NormalTok{clustering, k)}\SpecialCharTok{$}\NormalTok{tot.withinss)}
\CommentTok{\# distances currently list of lists this turns it into list}
\NormalTok{distances }\OtherTok{\textless{}{-}} \FunctionTok{unlist}\NormalTok{(distances, }\AttributeTok{recursive=}\ConstantTok{FALSE}\NormalTok{)}
\CommentTok{\# make distances a data frame and then plot k vs distance}
\NormalTok{distance\_df }\OtherTok{\textless{}{-}} \FunctionTok{data.frame}\NormalTok{(distances)}
\NormalTok{distance\_df}\SpecialCharTok{$}\NormalTok{ID }\OtherTok{\textless{}{-}} \FunctionTok{seq.int}\NormalTok{(}\FunctionTok{nrow}\NormalTok{(distance\_df))}
\FunctionTok{ggplot}\NormalTok{(distance\_df, }\FunctionTok{aes}\NormalTok{(}\AttributeTok{x=}\NormalTok{ID, }\AttributeTok{y=}\NormalTok{distances)) }\SpecialCharTok{+} \FunctionTok{geom\_point}\NormalTok{() }\SpecialCharTok{+} \FunctionTok{xlab}\NormalTok{(}\StringTok{"k"}\NormalTok{) }\SpecialCharTok{+} \FunctionTok{ylab}\NormalTok{(}\StringTok{"Average Distance"}\NormalTok{) }\SpecialCharTok{+} \FunctionTok{ggtitle}\NormalTok{(}\StringTok{"Average Distances vs Number of Clusters"}\NormalTok{)}
\end{Highlighting}
\end{Shaded}

\includegraphics{Weeks-11---12-Exercises_files/figure-latex/unnamed-chunk-6-1.pdf}
Based on the above graph, the elbow point is at k=5, meaning 5 clusters
is an ideal amount for the dataset.

\end{document}
